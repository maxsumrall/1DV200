\documentclass[]{apa}
\usepackage{setspace}
\affiliation{1DV200}
\author{Jonathan Sumrall, Ryan Lengel}
\title{Practical Assignment 3 - Attitudes }
\begin{document}
\doublespace
\maketitle{}

Mobile computing by the way of smart phones is a way a large portion of computer users access their data and communicate with others. Mobile computers then present a new target for computer hackers to gain confidential information, as a way for accidents to occur and confidential information to leak out, and for people to misunderstand the accessibility of their information and let out more than they intend to. It's not initially clear what information should be confidential, and perhaps the answer varies greatly between people. Since mobile phones are the way many people access data, there are new opportunities to implement security at level that was previously unavailable. The sand-boxing techniques that the iOS platform incorporates and places applications in is an example of this. 

I propose to make a small study about the security awareness and concerns of smart phone users. The purpose would be know gain insight as to what users know about security in mobile phones. There is many things they do with their phones that they might not know that these actions are public, easily obtained, or guarded against. It might allude to what concerns them if they are asked about what they are aware of as far as security. Should users show high levels of awareness about a particular issue, then that might be an issue they are concerned with. For example, location information is possible with mobile phones. Users can be located within a small radius. If people are aware of this, perhaps they don't mind it being known. Or perhaps they do not realize this can happen. Then the question is if they are concerned about this information being told to another party. 

Knowledge about users awareness and concerns in mobile computing is useful for someone who may be writing software for mobile phones. Implementing security is an investment of time and money, and to invest in security in one aspect of the users experience would be a large waste if they are not concerned (or perhaps not aware) with this security. It could actually be a problem if users prefer to trade their privacy for ease of use. 

We propose to implement this study as a survey of students here at the campus. Students will be asked to answer a series of questions to test their knowledge and awareness, and then more questions to ask them what concerns them about data in their mobile phone. Our hypothesis is that students will be aware that they have confidential data on their phone and that it could be accessed, but that they will not know how much data could easily be accessed by third party applications. As far as their concerns about the security, we hypothesize that students will be open to allowing  third parties into their private data so long as they receive some services from the third party in return.

Students will be given a series of questions, and there will be no individual interviews. The questions will be available online, so we will not be able to verify the identity of the correspondents. 

\end{document}